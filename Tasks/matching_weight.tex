% Options for packages loaded elsewhere
\PassOptionsToPackage{unicode}{hyperref}
\PassOptionsToPackage{hyphens}{url}
%
\documentclass[
]{article}
\usepackage{amsmath,amssymb}
\usepackage{iftex}
\ifPDFTeX
  \usepackage[T1]{fontenc}
  \usepackage[utf8]{inputenc}
  \usepackage{textcomp} % provide euro and other symbols
\else % if luatex or xetex
  \usepackage{unicode-math} % this also loads fontspec
  \defaultfontfeatures{Scale=MatchLowercase}
  \defaultfontfeatures[\rmfamily]{Ligatures=TeX,Scale=1}
\fi
\usepackage{lmodern}
\ifPDFTeX\else
  % xetex/luatex font selection
\fi
% Use upquote if available, for straight quotes in verbatim environments
\IfFileExists{upquote.sty}{\usepackage{upquote}}{}
\IfFileExists{microtype.sty}{% use microtype if available
  \usepackage[]{microtype}
  \UseMicrotypeSet[protrusion]{basicmath} % disable protrusion for tt fonts
}{}
\makeatletter
\@ifundefined{KOMAClassName}{% if non-KOMA class
  \IfFileExists{parskip.sty}{%
    \usepackage{parskip}
  }{% else
    \setlength{\parindent}{0pt}
    \setlength{\parskip}{6pt plus 2pt minus 1pt}}
}{% if KOMA class
  \KOMAoptions{parskip=half}}
\makeatother
\usepackage{xcolor}
\usepackage[margin=1in]{geometry}
\usepackage{color}
\usepackage{fancyvrb}
\newcommand{\VerbBar}{|}
\newcommand{\VERB}{\Verb[commandchars=\\\{\}]}
\DefineVerbatimEnvironment{Highlighting}{Verbatim}{commandchars=\\\{\}}
% Add ',fontsize=\small' for more characters per line
\usepackage{framed}
\definecolor{shadecolor}{RGB}{248,248,248}
\newenvironment{Shaded}{\begin{snugshade}}{\end{snugshade}}
\newcommand{\AlertTok}[1]{\textcolor[rgb]{0.94,0.16,0.16}{#1}}
\newcommand{\AnnotationTok}[1]{\textcolor[rgb]{0.56,0.35,0.01}{\textbf{\textit{#1}}}}
\newcommand{\AttributeTok}[1]{\textcolor[rgb]{0.13,0.29,0.53}{#1}}
\newcommand{\BaseNTok}[1]{\textcolor[rgb]{0.00,0.00,0.81}{#1}}
\newcommand{\BuiltInTok}[1]{#1}
\newcommand{\CharTok}[1]{\textcolor[rgb]{0.31,0.60,0.02}{#1}}
\newcommand{\CommentTok}[1]{\textcolor[rgb]{0.56,0.35,0.01}{\textit{#1}}}
\newcommand{\CommentVarTok}[1]{\textcolor[rgb]{0.56,0.35,0.01}{\textbf{\textit{#1}}}}
\newcommand{\ConstantTok}[1]{\textcolor[rgb]{0.56,0.35,0.01}{#1}}
\newcommand{\ControlFlowTok}[1]{\textcolor[rgb]{0.13,0.29,0.53}{\textbf{#1}}}
\newcommand{\DataTypeTok}[1]{\textcolor[rgb]{0.13,0.29,0.53}{#1}}
\newcommand{\DecValTok}[1]{\textcolor[rgb]{0.00,0.00,0.81}{#1}}
\newcommand{\DocumentationTok}[1]{\textcolor[rgb]{0.56,0.35,0.01}{\textbf{\textit{#1}}}}
\newcommand{\ErrorTok}[1]{\textcolor[rgb]{0.64,0.00,0.00}{\textbf{#1}}}
\newcommand{\ExtensionTok}[1]{#1}
\newcommand{\FloatTok}[1]{\textcolor[rgb]{0.00,0.00,0.81}{#1}}
\newcommand{\FunctionTok}[1]{\textcolor[rgb]{0.13,0.29,0.53}{\textbf{#1}}}
\newcommand{\ImportTok}[1]{#1}
\newcommand{\InformationTok}[1]{\textcolor[rgb]{0.56,0.35,0.01}{\textbf{\textit{#1}}}}
\newcommand{\KeywordTok}[1]{\textcolor[rgb]{0.13,0.29,0.53}{\textbf{#1}}}
\newcommand{\NormalTok}[1]{#1}
\newcommand{\OperatorTok}[1]{\textcolor[rgb]{0.81,0.36,0.00}{\textbf{#1}}}
\newcommand{\OtherTok}[1]{\textcolor[rgb]{0.56,0.35,0.01}{#1}}
\newcommand{\PreprocessorTok}[1]{\textcolor[rgb]{0.56,0.35,0.01}{\textit{#1}}}
\newcommand{\RegionMarkerTok}[1]{#1}
\newcommand{\SpecialCharTok}[1]{\textcolor[rgb]{0.81,0.36,0.00}{\textbf{#1}}}
\newcommand{\SpecialStringTok}[1]{\textcolor[rgb]{0.31,0.60,0.02}{#1}}
\newcommand{\StringTok}[1]{\textcolor[rgb]{0.31,0.60,0.02}{#1}}
\newcommand{\VariableTok}[1]{\textcolor[rgb]{0.00,0.00,0.00}{#1}}
\newcommand{\VerbatimStringTok}[1]{\textcolor[rgb]{0.31,0.60,0.02}{#1}}
\newcommand{\WarningTok}[1]{\textcolor[rgb]{0.56,0.35,0.01}{\textbf{\textit{#1}}}}
\usepackage{graphicx}
\makeatletter
\def\maxwidth{\ifdim\Gin@nat@width>\linewidth\linewidth\else\Gin@nat@width\fi}
\def\maxheight{\ifdim\Gin@nat@height>\textheight\textheight\else\Gin@nat@height\fi}
\makeatother
% Scale images if necessary, so that they will not overflow the page
% margins by default, and it is still possible to overwrite the defaults
% using explicit options in \includegraphics[width, height, ...]{}
\setkeys{Gin}{width=\maxwidth,height=\maxheight,keepaspectratio}
% Set default figure placement to htbp
\makeatletter
\def\fps@figure{htbp}
\makeatother
\setlength{\emergencystretch}{3em} % prevent overfull lines
\providecommand{\tightlist}{%
  \setlength{\itemsep}{0pt}\setlength{\parskip}{0pt}}
\setcounter{secnumdepth}{-\maxdimen} % remove section numbering
\ifLuaTeX
  \usepackage{selnolig}  % disable illegal ligatures
\fi
\IfFileExists{bookmark.sty}{\usepackage{bookmark}}{\usepackage{hyperref}}
\IfFileExists{xurl.sty}{\usepackage{xurl}}{} % add URL line breaks if available
\urlstyle{same}
\hypersetup{
  pdftitle={Practice on Causal Inference},
  pdfauthor={Lúa Arconada and Alejandro Macías},
  hidelinks,
  pdfcreator={LaTeX via pandoc}}

\title{Practice on Causal Inference}
\author{Lúa Arconada and Alejandro Macías}
\date{2024-01-06}

\begin{document}
\maketitle

\hypertarget{introduction-a}{%
\subsection{\texorpdfstring{Introduction
(\(A\))}{Introduction (A)}}\label{introduction-a}}

The data set contains information about some individuals of a given
species of animals. More specifically, it contains three variables:
their weight (in pounds), their illness status (sick or not sick), and
their chronological age (in years).

Our objective is to provide an estimation of the causal effect of a
relevant variable on a suitable response variable using the third
variable as possible confounding using the library \texttt{Matching}.

\hypertarget{method-m}{%
\subsection{\texorpdfstring{Method
(\(M\))}{Method (M)}}\label{method-m}}

We are going to use \texttt{Match} with the variable \texttt{weight} as
the response variable, \texttt{sick} as the intervention variable and
\texttt{age} as the potential confounder. In summary, we are going to
explore the question: How does being sick influence an individual's
weight?

This setup allows us to explore how being sick influences the weight of
the animals while controlling for the potential confounding effect of
their age. The basic idea is that if we match treated and untreated by
all confounders, the difference could be ascribed to a causal effect due
to sickness. Afterwards, we are going to check if matching is
appropiately balanced for the confounders.

\hypertarget{results-a-1}{%
\subsection{\texorpdfstring{Results
(\(A^{-1}\))}{Results (A\^{}\{-1\})}}\label{results-a-1}}

First, we carry out the matching process and obtain the following
output:

\begin{itemize}
\tightlist
\item
  Estimate: 69.659.
\end{itemize}

This is the estimated Average Treatment Effect on the Treated (ATT). In
the context of our data, it measures the average change in the outcome
variable (\texttt{weight}) for those sick compared to the control group
after matching on the covariate \texttt{age}. It indicates that, on
average, those who are \texttt{sick} have an increase of approximately
69.659 units in the \texttt{weigth} variable compared to the control
group.

\begin{itemize}
\tightlist
\item
  AI SE (Asymptotic Influence Standard Error): 31.2.
\end{itemize}

This is the standard error associated with the estimate. In simpler
terms, it tells us how much the estimate might vary if we were to repeat
the study. It is 31.2, suggesting considerable uncertainty around the
estimated treatment effect.

\begin{itemize}
\tightlist
\item
  T-stat (t-statistic): 2.2327.
\end{itemize}

The t-statistic is calculated by dividing the estimated treatment effect
by its standard error. It is used to test the null hypothesis that the
true treatment effect is zero. A t-statistic close to zero suggests that
the estimated effect is not significantly different from zero. Here, a
t-statistic of 2.2327 indicates that the estimated effect is about
2.2327 times the standard error away from zero.

\begin{itemize}
\tightlist
\item
  p.val (p-value): 0.025572.
\end{itemize}

The p-value is associated with the t-statistic. It indicates the
probability of observing a t-statistic as extreme as the one calculated,
assuming the null hypothesis (no treatment effect) is true. A small
p-value suggests evidence against the null hypothesis. Here, the p-value
is 0.025572, meaning that the effect of being sick on the weight of the
animals is statistically significant at the conventional 0.05
significance level.

\begin{itemize}
\tightlist
\item
  Original number of observations: 10.
\end{itemize}

The total number of observations in our original dataset.

\begin{itemize}
\tightlist
\item
  Original number of treated obs: 6.
\end{itemize}

The number of observations that received the treatment (e.g., `sick' =
1) in our dataset. In this case, there are 6 observations that are
`sick' in our dataset.

\begin{itemize}
\tightlist
\item
  Matched number of observations: 6.
\end{itemize}

The number of observations used in the matched sample after propensity
score matching.

\begin{itemize}
\tightlist
\item
  Matched number of observations (unweighted): 6.
\end{itemize}

The number of observations in the matched sample without considering
weights. In matching, sometimes weights are assigned to observations,
but here, the unweighted number is also provided.

In summary, the estimated effect suggests an increase in the `weight'
variable for those who are sick compared to healthy individuals. The
effect caused by sickness is statistically significant at the
conventional significance level of 0.05.

And secondly, we check if the matching is balanced, that is, if there
exists a sufficient number of treated and untreated (sick and healthy)
units in the dataset with similar weights. This is important since it is
desirable to have for each sick unit at least a healthy unit of similar
age, in order to best infer whether the effect on the weight is due to
the sickness status or some other covariate. Through both the naive and
bootstrap Kolmogorov-Smirnov tests p-values (0.048 and 0.04), as well as
through the T-test p-value (0.026), we can see that the dataset does not
present a good balance in terms of units to match according to age, at
least at the usual 0.05 significance level. That is, more units would be
need for this causal analysis of the effect of sickness status on weight
to be more secure.

\hypertarget{bibliography}{%
\subsection{Bibliography}\label{bibliography}}

Lin, L., Sperrin, M., Jenkins, D.A. et al.~A scoping review of causal
methods enabling predictions under hypothetical interventions. Diagn
Progn Res 5, 3 (2021).

\hypertarget{appendix}{%
\subsection{Appendix}\label{appendix}}

\begin{Shaded}
\begin{Highlighting}[]
\NormalTok{A }\OtherTok{=} \FunctionTok{Match}\NormalTok{(df}\SpecialCharTok{$}\NormalTok{weight, }\AttributeTok{Tr=}\NormalTok{df}\SpecialCharTok{$}\NormalTok{sick, }\AttributeTok{X=}\NormalTok{df}\SpecialCharTok{$}\NormalTok{age, }\AttributeTok{estimand=}\StringTok{\textquotesingle{}ATT\textquotesingle{}}\NormalTok{, }\AttributeTok{M=}\DecValTok{1}\NormalTok{)}
\FunctionTok{summary}\NormalTok{(A)}
\end{Highlighting}
\end{Shaded}

\begin{verbatim}
## 
## Estimate...  69.659 
## AI SE......  31.2 
## T-stat.....  2.2327 
## p.val......  0.025572 
## 
## Original number of observations..............  10 
## Original number of treated obs...............  6 
## Matched number of observations...............  6 
## Matched number of observations  (unweighted).  6
\end{verbatim}

\begin{Shaded}
\begin{Highlighting}[]
\NormalTok{MB }\OtherTok{=} \FunctionTok{MatchBalance}\NormalTok{(df}\SpecialCharTok{$}\NormalTok{sick }\SpecialCharTok{\textasciitilde{}}\NormalTok{ df}\SpecialCharTok{$}\NormalTok{age, }\AttributeTok{match.out=}\NormalTok{A, }\AttributeTok{nboots=}\DecValTok{500}\NormalTok{)}
\end{Highlighting}
\end{Shaded}

\begin{verbatim}
## 
## ***** (V1) df$age *****
##                        Before Matching        After Matching
## mean treatment........     51.597             51.597 
## mean control..........      32.99             44.029 
## std mean diff.........     270.57             110.05 
## 
## mean raw eQQ diff.....     16.619             8.5608 
## med  raw eQQ diff.....     15.634             9.2663 
## max  raw eQQ diff.....     20.958              14.25 
## 
## mean eCDF diff........    0.45833            0.38095 
## med  eCDF diff........        0.5            0.33333 
## max  eCDF diff........    0.83333            0.83333 
## 
## var ratio (Tr/Co).....    0.45213                Inf 
## T-test p-value........   0.025567            0.03178 
## KS Bootstrap p-value..      0.032               0.01 
## KS Naive p-value......   0.047619           0.015152 
## KS Statistic..........    0.83333            0.83333
\end{verbatim}

\end{document}
